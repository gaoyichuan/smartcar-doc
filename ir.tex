\documentclass{article}
\usepackage{xeCJK}
\usepackage{graphicx}
\usepackage{caption}
\usepackage{subfigure}
\usepackage[colorlinks=true,
linkcolor=blue
]{hyperref}
\usepackage{listings}
\usepackage[outputdir=build]{minted}
\usepackage{amsmath}
\usepackage{lastpage}   %获得总页数
\usepackage{fancyhdr}
\pagestyle{fancy}
\lhead{}
\chead{}
\rhead{}
\lfoot{}
\cfoot{}
\rfoot{\thepage\ of \pageref{LastPage}}%当前页 of 总页数
\renewcommand{\headrulewidth}{0.4pt}%改为0pt即可去掉页眉下面的横线
\renewcommand{\footrulewidth}{0.4pt}%改为0pt即可去掉页脚上面的横线
\lhead{\includegraphics[height=0.8in]{28.png}}  %在此处插入logo.pdf图片
\setCJKfamilyfont{yy}{YouYuan}                             %幼圆  yy


\newcommand{\yy}{\CJKfamily{yy}}
\renewcommand{\baselinestretch}{1.6}


\graphicspath{{./img/ir/}{./img/}}

\begin{document}


\title{6-5 红外寻迹模块介绍}
\author{工程物理系学生科协技术部\quad 周宇}
\date{\today}
\maketitle
\tableofcontents
\newpage


\section{前言}
这个文档旨在介绍官方代码中红外寻迹模块部分的使用方法,同时为大家讲解红外寻迹模块的工作原理。

\section{原理图介绍}
\begin{figure}[h]
\centering % 使后面的内容居中
\includegraphics[width=.8\textwidth]{1.png}
\caption{红外模块原理图}
\label{红外模块原理图}
\end{figure}
红外模块的原理图如图\ref{红外模块原理图}所示,下面我们开始逐个分析图中的元件。

\subsection{发射接收一体的红外对管}
无色透明为发射管,通电后能够产生人眼不可见红外光(手机拍照可见);

黑色部分为接收管,内部的电阻会随着接收到红外光的多少而变化:由于黑色吸光,当红外发射管照射在黑色物体上时反射回来的光就较少,接收管接收到的红外光就较少,表现为电阻大;当照射在白色表面时发射的红外线就比较多,表现为接收管的电阻较小。

\subsection{LM393比较器}
\begin{figure}[h]
\centering % 使后面的内容居中
\includegraphics[width=.4\textwidth]{2.png}
\caption{比较器}
\label{比较器}
\end{figure}
比较器有两个输入端和一个输出端,两个输入端一个称为同相输入端,用“+”号表示,另一个称为反相输入端,用“-”表示。用作比较两个电路时,任意一个输入端加一个固定电压作参考电压(也叫门限电压),另一端则直接接需要比较的信号电压。当“+”端电压高于“-”端电压时,输出正电源电压,当“-”端电压高于“+”端电压时,输出负电源电压(注意,此处所说的正电源电压和负电源电压是指接在比较器正负极的电压)。可以简单理解为:
\[
V_0=
\begin{cases}
1,  if V_{+}>V_{-}\\
0,  if V_{+}<V_{-}
\end{cases}
\]

如原理图所示,将红外接收端的电压信号传递给LM393的同相输入端“+”,这个变化的电压信号与电压比较器的反相输入端“-”的基准电压相比较,当同相端“+”的电压大于反相端“-”的电压时,电压比较器的输出端“DO”输出高电平电压,当同相端“+”电压小于反相端“-”端电压时,电压比较器的输出端“DO”输出低电平电压,此时开关指示灯亮。

在照射到黑色物体时,接收管的电阻值很大,同相输入端“+”电压升高,使同相端“+”电压大于反相端“-”端电压,电压比较器的输出端DO输出高电平电压,此时开关指示灯不亮。

在照射到白色物体时,接收管的电阻值很小,同相输入端“+”电压降低,使同相端“+”电压小于反相端“-”端电压,电压比较器的输出端DO输出低电平电压,此时开关指示灯亮。

接在反相端“-“端的电位器VR1用于调节该端的电位电压,这个电压也就是电压比较器输入的阀值翻转电压V-。调节红外模块的灵敏度时,仅需调节电位器来改变V-的值。使用时,通常将模块置于白色赛道上,微调电位器,当开关指示灯从熄灭刚好变为常亮即可。

(注:
一、运算放大器(简称“运放”)和比较器在电路图上的符号相同,如上图,要结合其他信息加以区分。简单讲,比较器是运放的开环应用,但比较器是针对电压门限比较而设计的,要求的比较门限要精确,一般情况下,用运放作比较器达不到满幅输出,而且翻转速度变慢(从ns变为us级),所以尽量不要用运放作比较器。

二、运放可以接入负反馈电路,而比较器不能。所以不能用比较器替换运放。)

\subsection{104电容(0.1uf)}
104电容并联在电源两段,作用是给电源滤波,大致原则为:系统频率越高,用的电容越小。一般10MHz的用104(0.1μF),100MHz的系统用103(0.01μF)。

另外,为了滤除高频干扰,提高电源稳定性,一般都要给每个芯片并联一个104的电容,并且越靠近芯片位置越好。

\section{红外模块使用方法}
分析完红外模块的原理,现在我们可以开始研究它如何使用了。

其实也非常简单,通过之前的原理分析,我们知道:在照射到黑色物体时,输出端DO输出高电平电压;在照射到白色物体时,输出端DO输出低电平电压。要想判断红外模块照射到的是黑色物体还是白色物体,用单片机检测一下D0是高电平还是低电平就好了。
\section{代码解析}
\lstset{language=C}
%%\begin{lstlisting}
\begin{minted}{c}
#include "IR.h"			//包含头文件IR.h
#define White OUTPUT_H	//宏定义:编辑源代码时,OUTPUT_H可以用White替换,
#define Black OUTPUT_L	//OUTPUT_L可以用Black替换,
		//HW_GPIO.h第395、396行定义OUTPUT_H为1,OUTPUT_L为0
#if USEIRQ	//见IR.h第15行,把0改为1,则启用中断,否则默认不使用中断
				//使用中断,则函数定义如下:
/****************************************
 * 函数名:IR_Init()
 * 功能描述:红外初始化函数
 * 参数:无
 * 返回:无
 ****************************************/
void IR_Init(){
  GPIO_InitTypeDef gpio_ir;		//定义结构体变量gpio_ir, 参见HW_GPIO.h第422-487行
  gpio_ir.GPIO_PTx=PTD;				//选择D端口的0-7八个引脚,见Port_Use.txt第28-41行
  gpio_ir.GPIO_Pins=GPIO_Pin0_7;
  gpio_ir.GPIO_PinControl=IRQC_ET;	//边沿触发外部中断,见HW_GPIO.h第416行
  gpio_ir.GPIO_Dir=DIR_INPUT;	//GPIO方向为“输入”
  gpio_ir.GPIO_Output=White;	//输出高电平
  gpio_ir.GPIO_Isr=IR_isr;		//调用红外中断回调函数,见IR.c第30-40行

  LPLD_GPIO_Init(gpio_ir);		//初始化gpio_ir,见HW_GPIO.c第29-117行


  LPLD_GPIO_EnableIrq(gpio_ir);	//使能GPIO外部中断,见HW_GPIO.c第119-154行
}

/****************************************
 * 函数名:IR_isr()
 * 功能描述:红外中断回调函数,选手可以修改
 * 参数:无
 * 返回:无
 ****************************************/
void IR_isr(){
  for(int i=0;i<8;i++)		//读取红外状态
    IR_state[i]=LPLD_GPIO_Input_b(PTD,i);	//HW_GPIO.c第353-374行
  }
}

#else		//不使用中断,则定义函数如下:
/****************************************
 * 函数名:IR_Init()
 * 功能描述:红外初始化函数
 * 参数:无
 * 返回:无
 ****************************************/
void IR_Init(){
  GPIO_InitTypeDef gpio_ir;	//定义结构体变量gpio_ir, 参见HW_GPIO.h第422-487行
  gpio_ir.GPIO_PTx=PTD;		//选择D端口的0-7八个引脚,见Port_Use.txt第28-41行
  gpio_ir.GPIO_Pins=GPIO_Pin0_7;
  gpio_ir.GPIO_Dir=DIR_INPUT;  //GPIO方向为“输入”
gpio_ir.GPIO_Output=White;	 //输出高电平

  LPLD_GPIO_Init(gpio_ir);//初始化gpio_ir,见HW_GPIO.c第29-117行
}

/****************************************
 * 函数名:IR_refresh()
 * 功能描述:红外刷新函数
 * 参数:无
 * 返回:无
 ****************************************/
void IR_refresh(){
  for(int i=0;i<8;i++){ 				//读取红外状态
    IR_state[i]=LPLD_GPIO_Input_b(PTD,i);  //HW_GPIO.c第353-374行
  }
}

#endif

%%\end{lstlisting}
\end{minted}

\section{函数应用举例}
\subsection{调用$IR\_Init()$函数来初始化红外传感器模块}

\lstset{language=C}
\begin{lstlisting}
void Overall_Init_All(){
  motor_InitializeMotor();
  OLED_Init();
  UART_Init();
  MPU6050_init();
  CCD_init();
  Led_Init();
  Button_Init();
  Timer_Init();
  IR_Init();
  Encoder_Init();
}
\end{lstlisting}
其中,$Overall\_Init\_All()$调用$IR\_Init()$来初始化红外传感器模块。

需要注意的是,所有模块的初始化函数都已经被包含在$Overall\_Init\_All()$中,只需要调用一次$Overall\_Init\_All()$就能完成全部的初始化工作。

\subsection{刷新8个红外模块的状态}
刷新红外状态的函数已经封装好,为$IR\_refresh()$,调用一次,当前当前8个红外模块的状态就会保存在全局变量数组$IR\_state$中。
\lstset{language=C}
\begin{lstlisting}
void main(void)
{
  Overall_Init_All();//初始化全部模块
  Draw_Startup();    //显示屏显示“工程物理系学生科协”
  while(1)
  {
		IR_refresh();		//调用IR_refresh()函数读取红外状态
		int a,b,i=0;
		while((IR_state[i]==White)&(i<8)) {i=i+1};
    	if (i==8) {a=0,b=0}   //全白,则a=0,b=0
		else{}
		motor_DirControlAB(a,b);		//控制AB电机方向
  }
}

\end{lstlisting}

上述代码只是红外刷新函数的一个简单示例,其实在主循环中持续刷新红外的状态是不明智的。这样会占用大量的运算资源,甚至会导致其他模块的工作异常,比较规范的使用方法是在定时中断中调用红外模块刷新函数,再根据红外模块的状态执行相应的策略,请参照《重要的辅助功能函数》中对于定时中断的介绍。

\subsection{启用红外模块中断}
我们用条件编译为选手准备好了另一种红外方案——启用红外模块中断。

要启用红外模块中断,我们首先要完成重要的第一步:

将IR.h中的\#define USEIRQ 0改为\#define USEIRQ 1

红外模块中断为上下沿触发中断,也就是说8个红外模块中任意一个的状态发生改变,程序就会开始执行红外模块的中断子程函数$IR\_isr()$,这个函数是提供给选手自己编辑的,官方代码中这个函数只有一行——刷新$IR\_state$。需要注意的是,这个函数是不能被选手调用的,它的调用是由单片机自动完成的。

假设我还想在中断子程中改变电机的方向,那我就只需改写这个中断子程函数$IR\_isr()$。

\lstset{language=C}
\begin{lstlisting}
void IR_isr(){
IR_refresh();		//调用IR_refresh()函数读取红外状态
motor_DirControlAB(1,1);		//控制AB电机方向
}

\end{lstlisting}
\end{document}
